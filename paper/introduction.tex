Computer Vision has historically been a very difficult research field to translate research improvements into useful commercial systems.  
This is partly because of the immense amount of data that CV solutions often need for training, and the relatively low level of accuracy that computer vision solutions often achieve.
However, in our opinion the results of certain classification tasks such as accuracy on CalTech101\cite{fei-fei:caltech101} is now reaching a level that CV can be incorporated into useful consumer products.

We propose in this work to use visual recognition on a novel dataset to develop a system that can help a tourist find their way around Columbia Campus.
We believe that this task is a feasible scenario in which a tourist would like to use our application to give a free tour to themselves based on simply snapping picture of their surroundings.
The idea and execution of this project is portable to other domains and locations, not just Columbia University.
It is highly feasible that this type of application could be useful in museums, zoos, historical sites, and other areas where there exists a constrained environment of the pictures that could possibly be taken.
We also propose extensions to simple visual search such as user interaction to actively grow our database of Columbia University building images, and a within category image clustering to constrain the memory usage of our application on the server side.

The layout of the paper is as follows.
In section 2, we discuss the collection of our novel dataset of Columbia University building images, and describe the dataset in detail.
In section 3, we detail our procedure which is greatly inspired by the work of Nister and Stewenius\cite{nister:vocabtree}.
In Section 4, we discuss the accuracy of the recognition framework and also discuss the speed with which we are able to accomplish the tasks.
Finally, In Section 5 we discuss our application architecture and the operation of our system.
