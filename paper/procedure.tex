Our object classification procedure can be broken down into 3 stages: feature
extraction, codebook generation, and k-d tree construction.

\subsection{Feature Extraction}
First, points of interest were detected in the image using OpenCV's StarDetector,
a variation of the CenSurE keypoint detector \cite{agrawal2008censure}. 
The CenSurE keypoint detector is a robust keypoint detector that is both fast and returned suitable points for image retrieval and matching.
Its detection speed over the SIFT detector was important in this application due to the need for quick response from our system to the user.
To extract features at these keypoints we use the well-known SIFT algorithm \cite{lowe1999object}. 
SIFT was chosen for its strong invariance under lighting conditions and scale which is important in the task of analyzing unpredictable user input. 
Additionally, these algorithms were chosen because they performed best in our initial experiments.
The results of some detected features within the images can be seen in 

\subsection{Codebook Generation}
To quantize the SIFT feature space, a codebook with k words was constructed.
Once we had a set of SIFT feature vectors for each image in our training set, k-means was applied to determine k cluster centers in the vector space. 
This set of cluster centers in the SIFT feature space was used as the codebook. 
A kd-tree was then constructed using OpenCV's FLANN \cite{muja2009fast} package to index the codebook so that it could be queried quickly. 
At this point, we had a data structure to efficiently map a SIFT feature vector to a code word.

We experimented using different sizes of codebooks within our experiments, and the results can be seen in Table \ref{tab:results}.
{\color{red} Maybe we could throw the kmeans optimization equation here... It's in the wikipedia article}.


\subsection{K-d Tree Construction and BoW Implementation}
Using this codebook, a histogram of word occurences was constructed for each training image. 
We use a FLANN k-d tree to index the features for each image into a histogram.
This is known as a Bag of Words image representation and is a popular and state of the art method for image retrieval and classifcation.
For each feature within the image we send the features down our k-d tree, and then sum the number of times a feature maps to a respective leaf node or code word for a respective image.
We are then left with an unnormalized histogram representation of features within each image.
We then normalize the images so that the sum of the histograms of each image is the same.
Finally, we implement a TF-IDF weighting scheme to maximize the information that is available within each image histogram.
TF-IDF is a commonly used technique in text processing to weight the important words within a document retrieval scheme.  
Words that appear in many documents, such as ``and'' and ``the'' receive very low weights, and words that are more subject and document specific receive higher weights in the classification scheme.
We use this idea for our image retreival system utilizing our learned codebook.
We weight each dimension of our BoW histogram representation using the equation \ref{eq:TF-IDF}.
%\begin{equation}
%w_{i} = \ln \frac{N}{N_{i}
%\end{equation}

FLANN was used again to construct an index of these histograms. 
We now have a data structure to efficiently map an unknown histogram to a known histogram.
In both vector spaces, euclidean distance is used as the distance metric. 
This decision was made based on experimentation.

Given a test image, the classification process consists of the following steps:
\begin{enumerate}
\item Extract the set of SIFT features (vectors of length 128).

\item Map SIFT features to pre-defined codewords

\item Bin the extracted SIFT features together to create a histogram representation of an image

\item Find the ``k'' closest matching histograms in the training set using euclidean distance.

\item Return the class of the matching histograms.
\end{enumerate}

In the future, there are several improvements we would like to make to this algorithm. 
In particular, the classifier would be more scalable if the word frequency histogram space were quantized using a codebook like the SIFT feature space. 
An intuitive way to think about this is that one only needs a few images of an object from different angles to recognize its most striking features.
Our guess is that if we found 8 histogram cluster centers for each building, they would be from pictures taken from different perspectives, capturing necessary features while .



